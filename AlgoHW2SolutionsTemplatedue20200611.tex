% me=0 student solutions (ps file), me=1 - my solutions (sol file),
% me=2 - assignment (hw file)
\def\me{0} \def\num{2} %homework number

\def\assigned{June 4, 2020} %due date
\def\due{5 pm on Thursday, June 11, 2020} %due date

\def\course{CSCI-GA.1170-001/002 Fundamental Algorithms} 
%course name, changed only once

% **** INSERT YOUR NAME HERE ****
\def\name{Gina Holden}

% **** INSERT YOUR NETID HERE ****
\def\netid{gh1407}

% **** INSERT NETIDs OF YOUR COLLABORATORS HERE ****
\def\collabs{NetID1, NetID2}


\iffalse

INSTRUCTIONS: replace # by the homework number.  (if this is not
ps#.tex, use the right file name)

Clip out the ********* INSERT HERE ********* bits below and insert
appropriate LaTeX code.  There is a section below for student macros.
It is not recommended to change any other parts of the code.


\fi
%

\documentclass[11pt]{article}


% ==== Packages ====
\usepackage{amsfonts}
\usepackage{latexsym}
\usepackage{fullpage}
\usepackage{amsmath}

% \setlength{\oddsidemargin}{.0in} \setlength{\evensidemargin}{.0in}
% \setlength{\textwidth}{6.5in} \setlength{\topmargin}{-0.4in}
\setlength{\footskip}{1in} \setlength{\textheight}{8.5in}

\newcommand{\commentt}[1]{}

\newcommand{\handout}[5]{
\renewcommand{\thepage}{#1, Page \arabic{page}}
  \noindent
  \begin{center}
    \framebox{ \vbox{ \hbox to 5.78in { {\bf \course} \hfill #2 }
        \vspace{4mm} \hbox to 5.78in { {\Large \hfill #5 \hfill} }
        \vspace{2mm} \hbox to 5.78in { {\it #3 \hfill #4} }
        \ifnum\me=0
        \vspace{2mm} \hbox to 5.78in { {\it Collaborators: \collabs
            \hfill} }
        \fi
      } }
  \end{center}
  \vspace*{4mm}
}

\newcounter{pppp}
\newcommand{\prob}{\arabic{pppp}} %problem number
\newcommand{\increase}{\addtocounter{pppp}{1}} %problem number

% Arguments: Title, Number of Points
\newcommand{\newproblem}[2]{
  \ifnum\me=0
    \ifnum\prob>0 \newpage \fi
    \increase
    \setcounter{page}{1}
    \handout{\name{} (\netid), Homework \num, Problem \arabic{pppp}}
    {\assigned}{Name: \name{} (\netid)}{Due: \due}
    {Solutions to Problem \prob\ of Homework \num\ (#2)}
    \section*{Problem \num-\prob~(#1) \hfill {#2}}
  \else
    \increase
    \section*{Problem \num-\prob~(#1) \hfill {#2}}
  \fi
}

% \newcommand{\newproblem}[2]{\increase
% \section*{Problem \num-\prob~(#1) \hfill {#2}}
% }

\def\squarebox#1{\hbox to #1{\hfill\vbox to #1{\vfill}}}
\def\qed{\hspace*{\fill}
  \vbox{\hrule\hbox{\vrule\squarebox{.667em}\vrule}\hrule}}
\newenvironment{solution}{\begin{trivlist}\item[]{\bf Solution:}}
  {\qed \end{trivlist}}
\newenvironment{solsketch}{\begin{trivlist}\item[]{\bf Solution
      Sketch:}} {\qed \end{trivlist}}
\newenvironment{code}{\begin{tabbing}
    12345\=12345\=12345\=12345\=12345\=12345\=12345\=12345\= \kill }
  {\end{tabbing}}

%\newcommand{\eqref}[1]{Equation~(\ref{eq:#1})}

\newcommand{\hint}[1]{({\bf Hint}: {#1})}
% Put more macros here, as needed.
\newcommand{\room}{\medskip\ni}
\newcommand{\brak}[1]{\langle #1 \rangle}
\newcommand{\bit}[1]{\{0,1\}^{#1}}
\newcommand{\zo}{\{0,1\}}
\newcommand{\C}{{\cal C}}

\newcommand{\nin}{\not\in}
\newcommand{\set}[1]{\{#1\}}
\renewcommand{\ni}{\noindent}
\renewcommand{\gets}{\leftarrow}
\renewcommand{\to}{\rightarrow}
\newcommand{\assign}{:=}

\newcommand{\AND}{\wedge}
\newcommand{\OR}{\vee}

\newcommand{\For}{\mbox{\bf for }}
\newcommand{\To}{\mbox{\bf to }}
\newcommand{\Do}{\mbox{\bf do }}
\newcommand{\If}{\mbox{\bf if }}
\newcommand{\Then}{\mbox{\bf then }}
\newcommand{\Else}{\mbox{\bf else }}
\newcommand{\While}{\mbox{\bf while }}
\newcommand{\Repeat}{\mbox{\bf repeat }}
\newcommand{\Until}{\mbox{\bf until }}
\newcommand{\Return}{\mbox{\bf return }}
\newcommand{\Halt}{\mbox{\bf halt }}
\newcommand{\Swap}{\mbox{\bf swap }}
\newcommand{\Ex}[2]{\textrm{exchange } #1 \textrm{ with } #2}




\begin{document}

\ifnum\me=0

% Collaborators (on a per task basis):
%
% Task 1: *********** INSERT COLLABORATORS HERE *********** 
% Task 2: *********** INSERT COLLABORATORS HERE *********** 
% etc.
%

\fi

\ifnum\me=1

\handout{PS \num}{\assigned}{Lecturer: Richard Hull}{Due: \due}
{Solution {\em Sketches} to Problem Set \num}

\fi

\ifnum\me=2

\handout{PS \num}{\assigned}{Lecturer: Richard Hull}{Due: \due}{Problem
  Set \num}

\fi


%======================================

\newproblem{Order of growth}{12 points}

\begin{itemize}
\item[(a)] (8 points) Write the following functions in the
$\Theta$-notation in the following simple form $c^n \cdot n^d \cdot (\log{n})^r$, where $c,d,r$ are constants. E.g., $20n^2+5n+7 \sim n^2$, you can omit ``$\Theta$'' in front of $n^2$. Briefly justify your answers
$$
\begin{array}{c}
\left(\log_5 n\right)^{1+1/\log_2(\log_5(n))},~
\log_{10}(n!) + 3 n^3,~
81^{\frac{1}{\log_n{\sqrt{3}}}} + 22 ,~
7n^{\frac{4}{3}} + 3 n 
\vspace*{.1in}\\
n^{\frac{n}{\log_2 n}+\frac{n}{\log_3 n}+\frac{n}{\log_4 n}}
,~
\frac{7n^3}{2} + 3 n ,~
n^{0.2} + 12(\lg n)^{24},~
15^{(3n - 7)/20}
\end{array}
$$

\ifnum\me<2
\begin{solution}   INSERT YOUR SOLUTION HERE   \end{solution}
\fi
\item[(b)] (4 points) List the simple functions you derived in part
  (a) in the order of asymptotic growth, from the smallest to the
  largest. (E.g., if one of the ``complicated'' functions in part (a)
  was $20n^2+5n+7$, use the equivalent ``simple'' function $n^2$ in
  your ordering for this problem).
\ifnum\me=2
	\fi
\ifnum\me<2
\begin{solution}   INSERT YOUR SOLUTION HERE   \end{solution}
\fi
\end{itemize}

%======================================

\newproblem{Recurrence examples
}{14 points}

Give asymptotic upper and lower bounds to $T(n)$
in each of the following recurrences.
Make your bounds as tight as possible and justify
your answers.
(For all of these, do not worry about rounding numbers to integers.)

\begin{itemize}
%-------------------------

\item[(a)] (2 point) 
$T(n) = T(n-3) + n^3$

\ifnum\me<2
\begin{solution}   INSERT YOUR SOLUTION HERE   \end{solution}
\fi

%-------------------------

\item[(b)] (2 points) 
$T(n) = 3T(n/3) + n^5$

\ifnum\me<2
\begin{solution}   INSERT YOUR SOLUTION HERE   \end{solution}
\fi

%-------------------------

\item[(c)] (2 points)
$T(n) = 7T(n/9) + \sqrt{n}$

\ifnum\me<2
\begin{solution}   INSERT YOUR SOLUTION HERE   \end{solution}
\fi




\item[(d)](4 points)
$T(n) = 2T(4\lfloor \sqrt[3]{n}\rfloor) + \lg^2n$.
(Recall that $\lg^2 n$ is defined as $(\lg n)^2$.)


\ifnum\me<2
\begin{solution}   INSERT YOUR SOLUTION HERE   \end{solution}
\fi

\end{itemize}

%======================================

\newproblem{Different Methods for Recurrences}{17 points}
Consider the following recurrence  $T(n) = 2 T(n/2) + n \log n$,
$T(1)=1$.

\begin{itemize}

\item[(a)] (2 points)
Can the master's theorem, as stated in the book, be applied to solve
this recurrence? If yes, apply it. If not, formally explain the reason
why.

\ifnum\me<2
\begin{solution}   INSERT YOUR SOLUTION HERE   \end{solution}
\fi

\item[(b)] (4 points)
Solve the above recurrence using the recursion tree method to get an asymptotically
tight bound for $T(n)$. 

\ifnum\me<2
\begin{solution}   INSERT YOUR SOLUTION HERE   \end{solution}
\fi

\item[(c)] (6 points)
Formally verify that your answer from part (b) is correct using
induction. \hint{Do not forget to prove it is a \emph{tight} bound.}

\ifnum\me<2
\begin{solution}   INSERT YOUR SOLUTION HERE   \end{solution}
\fi

\item[(d)] (5 points)
Solve the above recurrence exactly using domain-range substitution. 
Namely, make several changes of variables
until you get a basic recurrence of the form $$R(k) = R(k-1) + f(k)$$
for some $f$, and then compute the answer from there. Make sure you
carefully maintain the correct initial condition.

\ifnum\me<2
\begin{solution}   INSERT YOUR SOLUTION HERE   \end{solution}
\fi

\end{itemize}



%======================================

\newproblem{Multiplying large integers)
}{15 points}


\begin{itemize}
\item [(a)] (3 points)
Recall that if given two complex numbers $(a+bi)$ and $(c+di)$ then their product is $(ac - bd) + (bc + ad)i$.  For this exercise please find a way to multiply 2 complex numbers using only 3 multiplications and an arbitrary number of additions.  Justify the correctness of your solution.  How many additions are you using?
(Hint: play with terms that are variations of
$(a \pm b)(c \pm d)$)

\ifnum\me<2
\begin{solution}   INSERT YOUR SOLUTION HERE   \end{solution}
\fi

\end{itemize}

For the rest of this problem we focus on multiplying large base 10 integers.  First, recall the method that most of you learned in grade school, as illustrated here to compute $Z = XY$, where $X = 123$ and $Y= 789$.

\vspace*{.2in}
\noindent
{\bf Algorithm:} {\sc Classic Multiplication}
(by example)

\begin{description}
\item[Step 1:] Compute the single digit products, namely: 
$7X = 861$; $8X= 984$; $9X=1,087$
\item[Step 2:]
Left-Shift these products: 
$700X = 86,100$; $80X= 9,840$; $9X=1,087$

\item[Step 3:]
Add the aligned products: to get $97,047$
\end{description}

\begin{itemize}
\item[(b)] (3 points)
What is the asymptotic complexity of this algorithm, for base 10 integers both having the same length?  (For this, assume that additions and left shift operations take time $\Theta(n)$ for integers and shift amounts up to length $n$, and that 
multiplying 2 digits takes time $\Theta(1)$. 
In particular, then, multiplying a single-digit number and a number with $n$ digits takes $\Theta(n)$.)

\ifnum\me<2
\begin{solution}   INSERT YOUR SOLUTION HERE   \end{solution}
\fi

\end{itemize}
%-------------------------


Suppose now that $X$ and $Y$ have length $n$ where $n$ is a power of 2.
Consider the following Divide-and-Conquer approach for multiplication.
\ifnum\me<2
\newpage
\fi



\vspace*{.2in}
\noindent
{\bf Algorithm:} {\sc Simple Divide-and-Conquer}


\begin{description}
\item[Step 1:]
Split $X$ and $Y$ into two halves
$$
X = X_0 + 10^{(n/2)}X_1  \ \ \ \ \ \ 
Y = Y_0 + 10^{(n/2)}Y_1
$$
where $X_0$ is the lower-order half of $X$ and $X-1$ is the higher-order half of $X$, and similarly for $Y$.  (So, if $X = 0123$ then 
$X_1 = 01$ and $X_0 = 23$.)
Observe that
$$
\begin{array}{lcl}
Z \ = \ XY & = & (X_0 + 10^{(n/2)}X_1)(Y_0 + 10^{(n/2)}Y_1) \\
& = & X_0Y_0 + 10^{(n/2)}(X_0Y_1 + X_1Y_0) + 10^nX_1Y_1
\end{array}
$$
\item[Step 2:] Recursively compute:
$$
Z_0 := X_0Y_0, \ \ 
Z_{01} := X_0Y_1, \ \ 
Z_{10} := X_1Y_0, \ \ 
Z_2 := X_1Y_1
$$
\item[Step 3:]
Combine to form: 
$Z = Z_0 + 10^{(n/2)}(Z_{01} - Z_{10}) + 10^nZ_2$
\end{description}

\begin{itemize}
 \item[(c)] (2 points)
Write a recurrence that describes the running time of this algorithm.

\ifnum\me<2
\begin{solution}   INSERT YOUR SOLUTION HERE   \end{solution}
\fi

%-------------------------

\item[(d)] (3 points)
Solve the recurrence of part (c) to obtain the 
running time of that algorithm.

\ifnum\me<2
\begin{solution}   INSERT YOUR SOLUTION HERE   \end{solution}
\fi

\end{itemize}
%-------------------------

Karatsuba developed an improved Divide-and-Conquer for multiplication in 1962.  Here are the steps for his algorithm.

\vspace*{.2in}
\noindent
{\bf Algorithm:} {\sc Karatsuba}

\begin{description}
\item[Step 1:] Split $X$ and $Y$ as in preceding algorithm
\item[Step 2:]
Recursively compute the following 3 multiplications:
$$
Z_0 := X_0Y_0, \ \ \ 
z_2 := X_1Y_1, \ \ \ 
Z_1 := (X_0 + X_1)(Y_0 + Y_1)
$$
\item[Step 3:]
Combine these with the following formula:
$$
Z \ = \ Z_0 + 10^{(n/2)}(Z_1 - Z_0 - Z_2) + 10^nZ_2
$$
\end{description}

\begin{itemize}
\item[(e)] (3 points)  Write a recurrence that describes the running time of this algorithm

\ifnum\me<2
\begin{solution}   INSERT YOUR SOLUTION HERE   \end{solution}
\fi

%-------------------------

\item[(f)](4 points) Solve the recurrence of part (e) to obtain the asymptotic running time of Karatsuba's algorithm.  Include an explanation of how you solved it.

\ifnum\me<2
\begin{solution}   INSERT YOUR SOLUTION HERE   \end{solution}
\fi




\end{itemize}


\newproblem{Functionality vs. Running Time}{8 points}
In this question we explore the difference between
the functionality of a function and the running time
of the same. Recall that functionality refers to
the purpose of the procedure, i.e, what is it
designed to produce as an output. 
Consider the following recursive procedure.
\medskip

\begin{code}
  \>{\sc Bla}$(n)$:\\
  \> \> \If $n=1$ \Then \Return $4$\\
  \> \> \Else \Return $\mbox{\sc Bla}(n/2) + \mbox{\sc Bla}(n/2)+\mbox{\sc Bla}(n/2)+\mbox{\sc Bla}(n/2)$
\end{code}

\begin{itemize}

  \item[(a)] (3 points) What function of $n$ does {\sc Bla} compute
  (assume it is always called on $n$ which is a power of $2$)? This needs to be an
  exact answer and cannot an asymptotic bound. 
  \ifnum\me<2
\begin{solution}   INSERT YOUR SOLUTION HERE   \end{solution}
  \fi

  \item[(b)] (3 points) What is the running time $T(n)$ of {\sc Bla}
  (assuming the $\If\!$ statement and the addition can be accomplished
  in constant time)?

  \ifnum\me<2
\begin{solution}   INSERT YOUR SOLUTION HERE   \end{solution}
  \fi

  \item[(c)] (2 points) How do the answers to (a) and (b) change if
  the last line is replaced by\\ ``\Else \Return
  $4\cdot \mbox{\sc Bla}(n/2)$''?

  \ifnum\me<2
\begin{solution}   INSERT YOUR SOLUTION HERE   \end{solution}
\fi

\end{itemize}





\end{document}